\section{Dodatak}
\pythonexternal[caption={Klasa koja zapisuje na Tensorboard},label=pytorch_summary_writer_complete]{code/pytorch_summary_writer_complete.py}
Klasa \pythoninline{Monitor} pomoću klase \pythoninline{SummaryWriter} zapisuje napredak učenja agenta na Tensorboard.
\newpage

\pythonexternal[caption={Standardno q-učenje},label=q_learning]{code/q_learning.py}
Metoda \pythoninline{prepare_for_episode} priprema agenta za novu epizodu i u ovom slučaju samo poništava vrijednosti koje su bile pohranjene u varijablama koje predstavljaju prethodno stanje i akciju. Nadalje metoda \pythoninline{observe_environment} dodaje stanje u tablicu ako je prvi put posjećena, poziva funkciju \pythoninline{_update_q_table} i ažurira varijable trenutnog stanja i stope istraživanja. U ovoj cijeloj klasi najzanimljivija je funkcija \pythoninline{_update_q_table} u kojoj se pomoću Bellmanove jednadžbe optimalnosti ažuriranja ćelija u tablici. Još među bitnim metodama je \pythoninline{choose_action}, koja ovisno o tome dali se istražuje ili iskorištava stanje donese odluku o sljedećoj akciji.
\newpage

\pythonexternal[caption={Pomoćna klasa igrači za stolom}, label=poker_table_players]{code/table_players.py}
Implementirana je stilom klasične povezane liste, s kojom se omogućuje zatvaranje kruga, odnosno, igrač nakon posljednoga je prvi. Dodatno na ovaj način nije potrebno pratiti tko je trenutno diler, jer je to uvijek glava liste. Za intuitivno korištenje ove klase u petljama potrebno je implementirati Pythonove metode \pythoninline{__iter__} i \pythoninline{__next__} koje se brinu i o tome da petlja završava s posljednjim igračem. Podaci koje klasa čuva za svakog igrača su: referenca na objekt igrača, ime klase od tog objekta, ime igrača, trenutni ulog, sveukupni ulog, bodove (na kraju svakog kruga ovisno o jačini ruke dobije se određeni broj bodova), da li je aktivan (u smislu trenutnog kruga, da li aktivno sudjeluje u ulaganju/povećavanju uloga itd.), posljednji doneseni potez, krajnja ruka i kojeg je tipa ta ruka. Ova klasa implementira povezanu listu tako da ima referencu na sljedećeg igrača (što olakšava držanje redoslijeda igrača te mijenjanje dilera).
\newpage

\pythonexternal[caption={Klasa za pronalaženje najjače ruke}, label=poker_strongest_final_hand_finder]{code/strongest_final_hand_finder.py}
Ova se klasa uglavnom sastoji od statičkih metoda i nema potrebe da bilo što drži za sebe. Prima proizvoljan niz karata i vraća kombinaciju karata koja predstavlja najjaču ruku. Uspoređuje ruku s najjačom prema najslabijom, tako da u trenutku kada se pronađe odgovarajuća ruka nije više potrebno dalje povjeravati nego se ju vrati. Nije potrebno inicijalizirati objekt ove klase nego se funkcije direktno pozivaju preko klase.

\newpage

\pythonexternal[caption={Primjer test jedinica}, label=python_test_case]{code/python_test_case.py}