\section*{Sažetak}
\label{sec:summary}
\addcontentsline{toc}{section}{\nameref{sec:summary}}
Cilj ovog rada je implementirati pametnog agenta koji pomoću pojačanog učenja nauči igrati igru Poker. Poker je popularna igra karata koja ima razne varijacije pravila igranja. U ovom radu je implementiran Limit Texas Hold'em turnir u Python programskom jeziku verzije 3.7.10. Svi igrači ove igre primaju na početku određen broj žetona, te je glavni cilj skupiti sve protivničke žetone. Kroz razne faze u igri svaki igrač ulaže svoje žetone, ili odustaje od trenutačnog kruga. U konačnoj fazi igrač s najjačom rukom skuplja uložene žetone. Općenito Texas Hold'em je varijacija pokera u kojemu svaki igrač ima 2 karte u ruci, 5 se karata okreće na stol. Na kraju svaki igrač kombinira svoju konačnu ruku od pet proizvoljnih karata, od onih koje ima u ruci i onih sa stola. 'Limit' se odnosi na unaprijed dogovorenu vrijednost uloga, odnosno povećavanja uloga. Mogućnost povećavanja uloga se se zaustavlja nakon trećeg puta. 'Turnir' znači da se igra sve dok jedan igrač ne skupi sve žetone. Za izradu agenta korišteni su razvojni okviri (eng. \textit{Frameworks}) PyTorch 1.3.1 za sastavljanje i rad s neuronskom mrežom te Tensorboard 2.3.0 za vizualizaciju napretka i promjene neuronske mreže kroz učenje.

\noindent
\textbf{Ključne riječi:} Pojačano učenje, Poker, PyTorch, Tensorboard, Umjetna inteligencija