\section*{Sažetak}
\label{sec:summary}
\addcontentsline{toc}{section}{\nameref{sec:summary}}
Poker je popularna kartaška igra koja ima razne varijacije pravila igranja. U ovome radu se implementirao 'Limit Texas hold'em tournament' u Python programskom jeziku verzije 3.7.3. Svi igrači ove igre primaju na početku određen broj žetona, te glavni cilj je skupiti sve protivničke žetone. Kroz razne faze u igri svaki igrač ulaže svoje žetone, ili odustaje od trenutačnog kruga. U konačnoj fazi igrač sa najjačon rukon skuplja uložene žetone. Općenito 'Texas hold'em' je varijacija pokera u kojemu svaki igrač ima 2 karte u ruci, 5 karte se okreću na stolu i svaki igrač na kraju kombinira svoju konačnu ruku od pet proizvoljnih karte iz svoje ruke i stola. 'Limit' se odnosi na unaprijed dogovorenu vrijednost uloga, odnosno povisivanje uloga. Nakon trećeg ponovnog povišenog uloga nije više moguće povisiti ulog. 'Tournament' znači igra se sve dok jedan igrač ne skupi sve žetone. Cilj ovog rada je implementirati pametnog agenta koji pomoću pojačanog učenja nauči igrati ovu igru. Za izradu agenta korišteni su razvojni okviri (eng. \textit{Frameworks}) PyTorch 1.3.1 za sastavljanje i rad sa neuronskon mrežon te Tensorboard 2.3.0 za vizualizaciju napredka i promjene neuronske mreže kroz trening.