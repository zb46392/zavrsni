\section{Uvod}
Ovaj rad opisuje izradu kartaške igre 'Limit Texas Hold'em Tournament' u Python programskom jeziku. Za izradu igre se nije koristila niti jedna biblioteka koja se ne nalazi u Pythonovom standardnom skupu biblioteka. Poker je igra koja spada pod nepotpuno informirane igre. To znači da igrač koji sudjeluje u igri ne zna sveukupno stanje o igri, tj., nema informaciju o protivničkim kartama u ruci. Osim toga poker je specifičan jer uključuje i mogućnost sa lošom rukom pobijediti krug, na način praćenja i povisivanja uloga tako da se iskazuje lažni dojam protivnicima o posjedovanju dobre ruke, tzv. \emph{blef}. Također se u ovom radu opisuje izrada agenta koji pomoću metoda strojnog učenja vremenom poboljšava igranje igre. U svijetu strojnog učenja igre su uvijek imale ključnu ulogu već od samih početaka kada se postavlja temelj umjetne inteligencije. Ključni dio agenta je neuronska mreža koja prima stanje igre kao ulazni parametar u mrežu, a na izlazu svakoj akciji dodijeli vrijednost, te akcija sa najvišom vrijednosti se smatra potezom prikladnim za trenutačno stanje. Za izradu i upravljanje neuronske mreže koristio se PyTorch, kojeg je bilo potrebno dodatno instalirati. 

Glavni motiv ovog rada je dokazati da algoritam iz područja strojnog učenja može biti u stanju uspješno savladati jednu složenu igru, koja osim toga uključuje i određenu količinu nasumičnosti kao što je poker. Ovaj se problem pokušava riješiti pomoću neuronske mreže koja je trenutačno u centru pažnje umjetne inteligencije. 