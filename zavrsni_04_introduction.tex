\section{Uvod}
Ovaj rad opisuje izrada kartaške igre 'Limit Texas Hold'em Tournament' u Python programskom jeziku. Za izradu igre se nije koristila niti jedna biblioteka koja se ne nalazi u Pythonovom standardnom skupu biblioteka. Poker je igra koja spada pod nepotpuno informirane igre. Znači igrač koji sudjeluje u igri nezna sveokupno stanje o igri, tj., nema informaciju o protivničkim kartama u ruci. Osim toga poker je specifičan jer uključuje i mogućnost sa lošom rukom pobijediti krug, na način praćenje i povisivanje uloga tako da se iskazuje lažni dojam protivnicima o posjedovanje dobre ruke, tzv. \emph{blef}. Također se u ovome radu opisuje izrada agenta koji pomoću metodama strojnog učenja vremenom poboljšava igranje igre. U svijetu strojnog učenja igre su uvijek imali ključnu ulogu već od samih početaka kada se postavlja temelj umjetne inteligencije. Ključni dio agenta je neuronska mreža koja prima stanje igre kao ulazni parametar u mrežu, a na izlazu svakoj akciji dodijeli vrijednost, te akcija sa najvišon vrijednosti se smatra potezom prikladnim za tranutačno stanje. Za izgradu i upravljanje neuronske mreže koristio se PyTorch, kojeg je bilo potrebno dodatno instalirati. 

Glavni motiv ovoga rada je dokazati da algoritam iz područja strojnog učenja može biti u stanju uspješno savladati jednu složenu igru, koja osim toga uključuje i određenu količinu nasumičnosti kao što je poker. Ovaj se problem pokušava riješiti pomoću neuronske mreže koja je trenutačnu u centru pažnje umjetne inteligencije. 