\section{Zaključak}
Područje umjetne inteligencije je jako zanimljivo područje koje je još u razvoju. Zbog toga još nema čvrstih pravila kojima se garantira dobar rezultat, nego postoji puno pristupa koje je potrebno prilagoditi za određeni problem, a za visoku razinu uspješnosti potrebno je pokušati više različitih pristupa ili različitih kombinacija pristupa. Upute za rješavanje neke problematike s umjetnom inteligencijom se mogu pronaći u raznim oblicima, no u takvim uputama se često radi sa vrlo ograničenom domenom, što bitno olakšava njihovu primjenu. Osim slaganja arhitekture modela za učenje, potrebno je uložiti i dosta vremena u pripremu podataka za učenje, jer ih se može dobiti i prikazati u raznim oblicima. Osim toga potrebno je pogoditi prikladne hiperparametre, za koje također nema čvrstih pravila, nego se radi na osnovi pokušaja i neuspjeha. Naravno vremenom, se dobije neka intuicija kako otprilike složiti neke stvari, ali i dalje se pokušavaju ostale kako bi ih se međusobno usporedilo. U ovom radu se vidjelo da uspješnost učenja uvelike ovisi o prikladnim hiperparametrima poput stope učenja ili arhitekture same neuronske mreže. Za bilo koju vrstu problema koja se mogu riješiti algoritmički bez umjetne inteligencije je obično brže koristiti algoritmičko riješenje nego pomoću umjetne inteligencije učiti novo rješenje. Što se tiče implementacije igre Texas Hold’em, ostaje još dosta prostora za daljnja istraživanja i poboljšanja. Jako je kompleksna igra koja kombinira sreću, poznavanje igre i procjenjivanje protivničke reakcije. Svakako je moguće složiti neuronsku mrežu koja uspješno igra ovu igru na visokoj razini. Pokazalo se da se s jednom jednostavnom arhitekturom može smisleno igrati, što je, u stvari, i sama bit ovog rada.