\section{Zaključak}
Područje umjetne inteligencije je jako zanimljivo, međutim nema čvrstih pravila kojima se garantira dobar rezultat, nego postoji puno metoda od kojih svaka služi za određenu stvar. Potrebno je složiti određenoj problematici određene načine, a za visoku razinu uspješnosti potrebno je pokušati više različitih načina i različitih kombinacija načina. Upute za rješavanje neke problematike s umjetnom inteligencijom se mogu pronaći u raznim oblicima, no u takvim uputama se radi sa sterilnim podacima, što bitno olakšava rad. Do trenutka da su podaci spremni za učenje potrebno je uložiti dosta vrimena, jer ih se može prikazati u raznim oblicima. Tako da osim slaganja arhitekture modela za učenje, dosta se hrva sa sirovim podacima. Osim toga potrebno je pogoditi zgodne hiperparametre, za koje također nema čvrstih pravila, nego se radi na osnovi pokušaja i neuspjeha. Naravno vremenom, se dobije neka intuicija kako otprilike složiti neke stvari, ali i dalje se pokušavaju ostale za usporedbu s time. Za bilo koju vrstu problema koja se može riješiti algoritmički bez umjetne inteligencije je bolje ih se pridržavati i koristiti nego ulaziti u istraživanje za postojeće rješenje. Što se tiče implementacije za Texas Hold'em ostaje još dosta prostora za daljnja istraživanja i poboljšanja. Jako je kompleksna igra koja kombinira sreću, poznavanje igre i procjenjivanje protivničke reakcije. Svakako je moguće složiti neuronsku mrežu koja uspješno igra ovu igru na visokoj razini. Pokazalo se da se s jednom jednostavnom arhitekturom može smisleno igrati, što je bila i biti ovega rada.