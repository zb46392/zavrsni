\section{Zaključak}
Područje umjetne inteligencije je jako zanimljivo, međutim nema čvrstih pravila kojima se garantira dobar rezultat, nego postoje puno metoda od kojih svaka služi za određenu stvar. Potrebno je složiti određenoj problematici određene načine, a za visoku razinu uspješnosti potrebno je pokušati više različitih načina i različite kombinacije načina. Upute za rješavanje neku problematiku sa umjetnom inteligencijom se mogu pronaći u raznim oblicima, no u takvim uputama se radi sa sterilnim podacima, što bitno olakšava rad. Do trenutka da su podaci spremni za učenje potrebno je uložiti dosta vrimena, jer ih se mogu prikazati u raznim oblicima. Tako da osim slaganje arhitekture modela za učenje, dosta se hrva sa sirovim podacima. Osim toga potrebno je pogoditi zgodne hiperparametre, za kojih također nema čvrstih pravila, nego se radi na osnovi pokušaj i neuspjeh. Naravno vremenom, dobije se neka intuicija kako od prilike složiti neke stvari, ali i dalje se pokušavaju ostale za usporedbu sa time. Za bilo koju vrstu problema koja se može riješiti algoritmičke bez umjetne inteligencije je bolje ih se pridržavati i koristiti nego ulaziti u istraživanje za postojeće rješenje. Što se tiče implementacije za texas hold'em ostaje još dosta prostora za daljna istraživanja i poboljšanja. Jako je kompleksna igra koja kombinira sriću, poznavanje igre i procjenivanje protivničkove reakcije. Svakako je moguće složiti neuronsku mrežu koja uspješno igra ovu igru na visokoj razini. Pokazalo se da sa jednom jednostavnom arhitekturom može smisleno igrati, što je bila i bit ovega rada.