\section{Umjetna inteligencija}
Je jako široko područje računalne znanosti. Bavi se proučavanjem i izrade algoritama koje nemaju explicitnih naredbi kako se ponašati u nekom trenutku i/ili okolini, nego kroz određene procese sami uspiju donijeti zaključke i odluke. Umjetna inteligencija se dijeli u puno pod grana, koje u prošlosti nisu uspješno međusobno surađivali. U današnje doba se iskazao potencijal umjetne inteligencije i sve više se ulaže u njezino istraživanje i razvoj. 

Umjetna inteligencija se djeli na više grana kao što su:
\begin{itemize}
	\item Rasuđivanje i rješavanje problema: Gdje se pokušava implementirati razbivanje problema u logičke korake do riješenja te postepenim izvršavanje koraka riješiti zadani problem. Ovakvi algoritmi se nisu iskazali korisnim u velikim problemima zbog kombinatorijske eksplozije, znači rastom problema algoritam exponencijalno usporava.
	
	\item Reprezentacija znanja: Ovo je glavno područje u istraživanju klasične umjetne inteligencije. Skuplja se znanja od nekog područja i pohranjuje se u neku bazu u kojoj se istaknu pojmovi i veze između njih. Neke od stvari koje bi se nalazili u takvoj bazi su: predmeti, svojstva, vrste i odnose između predmeta, situacija, događaje, stanja i vremena, uzroci i posljedice, znanje o znanje i još puno manje istraženih područja.
	
	\item Planiranje: Odnosi se na predviđanje nekih budućih događaja te donošivanje određenih odluka koje utječu na izhod.
	
	\item Učenje: Predstavlja algoritme koji is početka jako loše riješavaju zadatak, međutim skupljanjem iskustva vremenom su sve bolje.
	
	\item Obrada prirodnog jezika: Omogućuje strojevima razumjevanje ljudskih jezika.
	
	\item Percepcija: Sposobnost pomoću određenih senzora primit informacije o stvarnoj okolini te uspješno interpretiranje te informacije u svoje zadaće.
	
	\item Kretanje i rukovanje: Upravljanje mehaničkih udova kako bi riješilo neki zadatak u stvarnoj okolini. Smatra se posebno kompleksnim, te obuhvaća paradoksni pojam (Moravec's paradox) u kojemu je teže implementirati radnju koju mlado dijete može izvoditi bez ikakvih poteškoća kao npr: primit neki predmet u ruke i odložit ga na odrećeni položaj od radnje kojoj je odraslome čovjeku teško kao npr: jako dobro odigrati partiju šaha. Prizlazi iz činjenice da za razliku od šaha percepcija, kretanje i rukovanje su prirodnom selekcijom kroz miljunima godinama usađeni u čovjeka.
	
	\item Socijalna inteligencija: Uključuje poznavanje i prepoznavanje emocije i ljudskih međudjelovanja. Nekim algoritmima bi bilo vrlo korisno da uz pojmove teorije uključuje spoznaju ljudskih emotivnih stanja i motive za donijeti bolje odluke.
	
	\item Opća inteligencija: U prošlosti se pokušalo dizajnirati opće inteligentnog agenta koji pokriva široku ljudsku spoznaju. No odustalo se od toga, zbog preogromne količine inforamcije koja je kombinirana iz raznih područja. Danas se razvija 'usku' umjetnu inteligenciju koja je usredotočena u jedno područje i smatra se da opća umjetna inteligencija bi trebala ujedinit hrpu tih usredotočenih područja.
\end{itemize}

\subsection{Strojno učenje}
U ovome završnom radu se razradilo područje iz učenja. Pa tako i ovo područje se grana na više podpodručja. Među najpoznatijima su učenje sa nazorom (eng. \textit{supervised learning}) u kojemu algoritam dobije skup podataka iz kojih treba donijet neke zaključe te skup riješenja, odnosno definirane zaključke koje bi trebao donijeti. Tako da kroz nekog vremena koje je dodjeljeno za učenje, primi podatak, donese odluku i ovisno 
o tome koji je unaprijed definirani zaključak kojeg je trebao donijeti se prilagodi da u buduće donosi što ispravnije odluke. Druga grana se zove učenje bez nazora (eng. \textit{unsupervised learning}), gdje algoritam prima neki skup podataka i vremenom uspije pronaći razlike i sličnosti u podacima. Nema informacije o tome šta podatak predstavlja ali zna podatke svrstiti u svoje definirane kategorije. Izrađeni agent iz ovog projekta spada pod skupinom pojačano učenje (eng. \textit{reinforcement learning}). U ovoj grani algoritam djeluje u nekom okruženju pomoću određenih akcija, te u početku nema informacije o tome kako akcije utječu na okruženje. Te vremenom mora sam otkriti koje akcije u kojem trenutku donosi najveću nagradu.
